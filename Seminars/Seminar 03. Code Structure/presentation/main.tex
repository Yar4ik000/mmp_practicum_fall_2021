\documentclass[fleqn, xcolor=x11names, 11pt]{beamer}
\usetheme{MyAmsterdam} %тема\
\usecolortheme{default}
\usepackage[utf8]{inputenc}
\usepackage[russian]{babel}
\usepackage[OT1]{fontenc}

\usepackage{amsmath}
\usepackage{amsfonts}
%\usepackage{amssymb}
\usepackage{hyperref}
\usepackage{graphics, graphicx}
\usepackage{color}
\usepackage{hyperref}
\usepackage{enumerate}

% ДЛЯ НАПИСАНИЯ КОДА
\usepackage{minted}
\usemintedstyle{default}
\newminted[pcode]{python}{baselinestretch=1, fontsize=\small}
\newmintinline[pinline]{python3}{baselinestretch=1}

\setbeamertemplate{navigation symbols}{} % отключить навигационные значки
\setbeamertemplate{footline}{%
    \hspace{0.94\paperwidth}%
    \usebeamerfont{title in head/foot}%
    \insertframenumber\,/\,\inserttotalframenumber%
} % переопределить нижнуюю панель, убрать всё кроме номера слайда


%\usefonttheme[onlylarge]{structurebold} % названия и текст в колонтитулах выводится полужирным шрифтом.
\usefonttheme[onlymath]{serif}  % привычный шрифт для математических формул
%\setbeamerfont*{frametitle}{size=\normalsize,series=\bfseries} % шрифт заголовков слайдов
\usepackage[nopar]{lipsum} %для генерации большого текста


\usepackage{tikz}
\usetikzlibrary{arrows,positioning}
\usepackage{listings}
\lstset{language=Python}

\newcommand{\real}{\mathbb{R}}
\newcommand{\norm}{\mathop{\rm norm}\limits}
\newcommand{\softmax}{\mathop{\rm softmax}\limits}

\definecolor{beamer@blendedblue}{rgb}{0.037,0.366,0.75}

\title[Введение в Python]{\bfseries Занятие 3: Функции. Модули. Классы.}
%\author[Попов~А.\,С.]{Попов Артём Сергеевич}
\subtitle{Практикум на ЭВМ, осень 2021}
\institute[ВМК МГУ]{МГУ имени М. В. Ломоносова, факультет ВМК, кафедра ММП}
\date{13 сентября 2021}

\begin{document}

\begin{frame}
\maketitle
\end{frame} 


\begin{frame}[fragile]\frametitle{Организация кода}

Функции, модули, пакеты, классы устраняют необходимость вставлять в программу избыточные копии блоков одинакового кода.

\hfill

Грамотная организация кода позволяет:
\begin{itemize}
\item Повысить частоту использование кода 

\item Минимизировать избыточность кода

\item Упростить отладку кода

\item Упростить написание тестов 

\item Улучшить читаемость кода (в том числе не автором кода)

\item Улучшить логику кода
\end{itemize}



\end{frame}


\section{Функции}
\subsection*{}
\begin{frame}[fragile]\frametitle{Синтаксис}

Определение функции:

\begin{minted}{python}
>>> def f(x):
...    """This function does nothing"""
...    return x
\end{minted}

\hfill

У аргументов функции могут быть значения по умолчанию:
(такие аргументы всегда идут в конце)
\begin{pcode}
>>> def f(x, y, z=0):
...    return x + y * z
\end{pcode}

\hfill

При вызове функции могут использоваться именованные аргументы (всегда идут за позиционными):
\begin{pcode}
>>> f(1, 2)
>>> f(1, z=2, y=1)
\end{pcode}

\hfill
\end{frame}



\begin{frame}[fragile]\frametitle{Методы функции}
Всё в Python --- объект, функция тоже объект!

\hfill

У функции есть некоторые атрибуты:
\begin{pcode}
>>> f.__doc__
"This function does nothing"
>>> f.__dict__ # список атрибутов функции
{}
\end{pcode}

\hfill

Функция со счётчиком запусков:
\begin{pcode}
>>> def f(x):
>>>    """This function does nothing"""
...    if 'f_count' in f.__dict__:
...        f.f_count += 1
...    else:
...        f.f_count = 1
...    return x
\end{pcode}

\end{frame}


\begin{frame}[fragile]\frametitle{Значения по умолчанию: проблемы}

Не нужно использовать изменяемые объекты в качестве значений по умолчанию:
\begin{pcode}

>>> def function(list_of_items, my_set=set()):
...     final_list = list()
...     for item in list_of_items:
...         if item not in my_set:
...             my_set.add(item)
...             final_list.append(item)
...     return final_list
...
\end{pcode}
\pause{
\begin{pcode}
>>> my_list = [1, 2, 3]
>>> function(my_list)
[1, 2, 3]
>>> function(my_list)
[]
\end{pcode}
}
\end{frame}

\begin{frame}[fragile]\frametitle{Функции с переменным числом аргументов}
У функции может быть переменное число неименованных и именованных аргументов:
\begin{pcode}

>>> def f(*args, **kwargs):
...    if 'x' in kwargs:
...        print(kwargs['x'])
...    print(args)
...    print(kwargs)
...
>>> f(3, 1, 7, x=1, y=8)
1
(3, 1, 7)
{'x': 1, 'y': 8}
\end{pcode}
\end{frame}



\begin{frame}[fragile]\frametitle{Пример: функция минимума}

Функция минимума из двух элементов:
\begin{pcode}

>>> def min(x, y):
...     return x if x < y else y
\end{pcode}

\hfill

Функция минимума для произвольного числа аргументов:
\begin{pcode}
>>> def min(*args):
...     res = float('inf')
...     for arg in args:
...         if arg < res:
...             res = arg
...     return res
\end{pcode}

\hfill

\textcolor{red}{У такой реализации есть проблема:}
\pause{
\begin{pcode}
>>> min()
inf
\end{pcode}
}
\end{frame}

\begin{frame}[fragile]\frametitle{Ограничения на аргументы}

Более правильная реализация с обязательным аргументом:
\begin{pcode}
>>> def min(first, *args):
...     res = first
...     for arg in args:
...         if arg < res:
...             res = arg
...     return res
\end{pcode}

Аргументы могу быть строго именованными:
\begin{pcode}
>>> def f(x, *args, age):
...    if age < 18:
...        return 0
...    return x + sum(args) 
...
>>> f(1, 2, 20)
TypeError:
f() missing 1 required keyword-only argument: 'age'
\end{pcode}

\end{frame}

\begin{frame}[fragile]\frametitle{Преобразование контейнеров в аргументы}
\mintinline{python}{*} перед <<контейнером>> преобразует его в аргументы функции
\begin{pcode}
>>> my_list = [1, 2, 3]
>>> # две эквивалентные записи
>>> f(my_list[0], my_list[1], my_list[2])
>>> f(*my_list)
\end{pcode} 

\hfill

\mintinline{python}{zip()} соединяет несколько <<контейнеров>> в <<контейнер>> кортежей:
\begin{pcode}
>>> your_list = ['a', 'b', 'c']
>>> t = list(zip(my_list, your_list))
>>> t
[(1, 'a'), (2, 'b'), (3, 'c')]
\end{pcode} 

\hfill

Обратное преобразование с использованием \mintinline{python}{*}:
\begin{pcode}
>>> list(zip(*t))
[(1, 2, 3), ('a', 'b', 'c')]
\end{pcode} 

\end{frame}

\begin{frame}[fragile]\frametitle{Области видимости}
\hfill

\textbf{Область видимости} --- место, где определяются переменные и выполняется их поиск. 

\hfill

Правило LEGB --- последовательный поиск имени в 4~областях:
\begin{enumerate}
\item[L] --- local (все имена, которым присваиваются значения внутри функции)

\item[E] --- enclosing (локальные области объемлющих функций) 

\item[G] --- global (имена на верхнем уровне модуля)

\item[B] --- built-in (встроенные имена)
\end{enumerate}

\hfill

Явное изменение области видимости имени (L на E):
\begin{pcode}
>>> def f():
... nonlocal y     # global y (L на G)
... y += 1
\end{pcode}

\hfill

\end{frame}

\begin{frame}[fragile]\frametitle{Пример работы LEGB}
\begin{pcode}
>>> x = 50
>>> y = 1
>>> def enc_func(x):
...    y = 30
...    def func(y):
...        nonlocal x
...        print(x + y)
...    func(11)
...
>>> enc_func(5)
???
\end{pcode}
\pause{
\begin{pcode}
16
\end{pcode}
}
\end{frame}

\begin{frame}[fragile]\frametitle{Для присваиваний LEGB не работает}
Операция присваивания по умолчанию создаёт локальную переменную:

\begin{pcode}
>>> x = 17
>>> def f():
...    x += 5
UnboundLocalError: local variable 'x' referenced before
assignment
\end{pcode}
\end{frame}

\begin{frame}[fragile]\frametitle{Анонимные (lambda) функции}
Когда нужно передать функцию в качестве аргумента или сделать короткую операцию несколько раз:

\hfill

\begin{pcode}
lambda x: x ** 2 + x - 1
\end{pcode}
{\Large $$ \sim $$}
\begin{pcode}
>>> def f(x):
...     return x ** 2 + x - 1
\end{pcode} 

\hfill

Можно (но не нужно) присвоить переменной лямбду:
\begin{pcode}
>>>    f = lambda x: x ** 2
>>>    f(2)
4
\end{pcode}
\end{frame}

\begin{frame}[fragile]\frametitle{Применение lambda функций}
Использование анонимной функции в функции sorted:

\begin{pcode}
>>> l = [(1, 0), (1, 1), (1, 2), (0, 3), (2, 1)]
>>> # по умолчанию сортируется по первому элементу
>>> sorted(l)
[(0, 3), (1, 0), (1, 1), (1, 2), (2, 1)]

>>> # сортируется по второму элементу
>>> sorted(l, key=lambda x: x[1])
[(1, 0), (1, 1), (2, 1), (1, 2), (0, 3)]

>>> # сортируется по произведению
>>> sorted(l, key=lambda x: x[0] * x[1])
[(1, 0), (0, 3), (1, 1), (1, 2), (2, 1)]
\end{pcode}
\end{frame}

\begin{frame}[fragile]\frametitle{Рекомендации по функциям}
\begin{itemize}
	\item По умолчанию возвращается \pinline{None}, не нужно писать отдельно \pinline{return None}.
	\item Не увлекайтесь возможностям динамической типизации: ограничивайте типы аргументов и выходов функции.
	\item Не делайте очень много аргументов. Заводите отдельные структуры данных для хранения большого количества информации.
\end{itemize}
	
\end{frame}

\section{Модули}
\subsection*{}

\begin{frame}[fragile]\frametitle{Синтаксис}

Модуль --- файл с расширением .py

\hfill

Оператор \mintinline{python}{import} импортирует указанный модуль и создаёт на него ссылку в текущей области видимости:

\begin{pcode}
>>> import my_module
>>> my_module.function_in_module()
\end{pcode}

\hfill

С помощью оператора \mintinline{python}{as} можно изменить имя переменной, в которую будет записана ссылка на модуль:

\begin{pcode}
>>> import my_module as mm
\end{pcode}

\hfill

С помощью связки \mintinline{python}{from ... import} можно импортировать имя из другого модуля в текущую область видимости:

\begin{pcode}
>>> from my_module import function_in_module
\end{pcode}

\hfill

\end{frame}


\begin{frame}[fragile]\frametitle{Ещё о модулях}

Если в втором аргументе \mintinline{python}{from ... import} указать *, импортируются все глобальные имена модуля или все переменные, записанные в \mintinline{python}{__all__}

\hfill

У модуля есть некоторые атрибуты:

\begin{pcode}
>>> import my_module
>>> my_module.__name__
'my_module'
>>> my_module.__file__
'/home/user/Programs/my_module.py'
\end{pcode}

\hfill

У модулей, переданных на выполнение интерпретатору, \mintinline{python}{__name__} == \mintinline{python}{__main__}


\end{frame}


\begin{frame}[fragile]\frametitle{Импортирование модуля}

Что происходит при импортировании модуля:

\begin{enumerate}

\item Поиск файла модуля

(по умолчанию по директориям, находящимся в переменной  \mintinline{python}{sys.path})

\item Компиляция в байт-код

(если нет актуальной скомпилированной версии модуля)

\item Запуск программного кода модуля

\end{enumerate}

\hfill

Если в модуле что-то изменилось, его необходимо перегрузить:

\begin{pcode}
>>> import imp
>>> imp.reload(my_module)
\end{pcode}


\end{frame}

\begin{frame}[fragile]\frametitle{Магическая команда \texttt{\% autoreload}}
\begin{itemize}
	\item \texttt{\%autoreload 2} --- перезагружать все модули при выполнении кода (кроме прописанных в \texttt{\%aimport-foo})
    \item \texttt{\%autoreload 1} --- перезагружать все модули загруженные с помошью  \texttt{\%aimport}, при выполнении кода 
    \item \texttt{\%autoreload 0} --- отключить автоматическую перезагрузку
    \item \texttt{\%aimport foo} --- импортировать \texttt{foo} и включить для него автоматическую перезагрузку
    \item  \texttt{\%aimport -foo} --- выключить для \texttt{foo} перезагрузку
\end{itemize}
 
\hfill

\begin{pcode}
>>> %load_ext autoreload
>>> %autoreload 1
>>> %aimport module
>>> module.variable
5
>>> # изменили модуль
>>> module.variable
7
\end{pcode}

\end{frame}

\begin{frame}[fragile]\frametitle{Когда \texttt{\_\_name\_\_ == \_\_main\_\_}}

Модуль может содержать не только функции или классы, но и исполняющие их инструкции.

Чтобы эта часть кода не выполнялась при импорте, её заключают в специальный блок:

\begin{pcode}
def sin(x):
    return x - x ** 3 / 6

if __name__ == '__main__':
    print('test my method')
    print(sin(3.14))
    print(sin(0))
    print('end of test')
\end{pcode}

\end{frame}

\begin{frame}[fragile]\frametitle{От модулей к пакетам}
Пакет --- каталог, включающий в себя другие каталоги и модули, содержащий файл \texttt{\_\_init\_\_.py}.
	
\hfill
	
Пакеты позволяют организовать импорт <<через точку>>:

\begin{pcode}
>>> from numpy.random import randn
\end{pcode}
	
\hfill
		
Также пакеты позволяют организовать относительный импорт:
\begin{pcode}
>>> from .my_module import my_function
>>> from ..another_package.its_module import its_function
\end{pcode}

\hfill

Файл \texttt{\_\_init\_\_.py} может использоваться для изменения поведения импорта.
\end{frame}

\begin{frame}[fragile]\frametitle{Пример использования \texttt{\_\_init\_\_.py}}
Пусть файлы проекта устроены так:
\begin{verbatim}
package:
    __init__.py
    first_module.py
        def f1
        def f2
\end{verbatim}

\hfill

Файл \texttt{\_\_init\_\_.py} устроен так:
\begin{pcode}
>>> from .first_module import f1
\end{pcode}

\hfill

Тогда можно делать так:
\begin{pcode}
>>> from package import f1
\end{pcode}

\end{frame}

\begin{frame}{Рекомендации по модулям и пакетам}
	\begin{itemize}
		\item Всегда организовывайте код в модули, объединяйте логические связанные модули в пакеты
		\item В пакетах крайне желательно использовать относительный импорт
		\item Осторожно делайте перекрёстные импорты между пакетами и модулями, чтобы не получить циклический импорт
	\end{itemize}
\end{frame}
\section{Классы}
\subsection*{}

\begin{frame}[fragile]\frametitle{Синтаксис}
Объявление класса:
\begin{pcode}
>>> class MyClass:
...    def __init__(self, value):    # конструктор класса
...        self.value = value
...    
...    def get_value(self):
...        return self.value 
\end{pcode}

\hfill

При вызове класса порождается его экземпляр:
\begin{pcode}
>>> new_obj = MyClass(2)
\end{pcode}

\hfill

Каждый объект экземпляра наследует атрибуты класса и  приобретает
свое собственное пространство имен

\end{frame}

\begin{frame}[fragile]\frametitle{Атрибуты экземпляров класса}

Атрибут класса $\neq$ атрибут экземпляра

\hfill

Атрибуты экземпляра добавляются к экземпляру с помощью использования self или прямым присваиванием к экземпляру:  
\begin{pcode}
>>> class MyClass:
...    def __init__(self, value):
...        self.value = value
...    
>>> new_obj = MyClass(2)
>>> new_obj.obj_attr = -2
\end{pcode}

\hfill

\mintinline{python}{value} и \mintinline{python}{obj_attr} --- атрибуты экземпляра класса

\hfill

\end{frame}


\begin{frame}[fragile]\frametitle{Атрибуты класса}

Атрибут класса $\neq$ атрибут экземпляра

\hfill

Атрибуты класса объявляются в теле класса или прямым присваиванием к классу. Изменение атрибута класса отразится на всех его экземплярах:

\begin{pcode}
>>> class MyClass:
...    class_attr = 0
...    def __init__(self, value):
...        self.value = value
...    
>>> new_obj = MyClass(2)
>>> MyClass.second_class_attr = '127'
>>> new_obj.second_class_attr
'127'
\end{pcode}

\hfill

\mintinline{python}{class_attr} и \mintinline{python}{second_class_attr} --- атрибуты класса

\end{frame}

\begin{frame}[fragile]\frametitle{Связанные и несвязанные методы}
У связанного метода первый аргумент --- соответствующий экземпляр класса:
\begin{pcode}
>>> class MyClass:
...    def my_method(self):
...        print('Do nothing')
...
>>> new_obj = MyClass()
>>> new_obj.my_mehtod() # связанный метод
'Do nothing'
\end{pcode}

\hfill

Несвязанному методу необходимо явно передать первым аргументом экземпляр класса:
\begin{pcode}
>>> the_same_method = MyClass.my_method # несвязанный метод
>>> the_same_method()
TypeError: my_method() missing 1 required positional 
argument: 'self'
>>> the_same_method(new_obj)
'Do nothing'
\end{pcode}

\hfill

\end{frame}

\begin{frame}[fragile]\frametitle{Внутренние атрибуты классов и экземпляров}

\begin{pcode}
>>> class MyClass:
...    """Documentaion"""
...    def __init__(self, value):
...        self.value = value
...        
...    def one_method():
...        print('Do nothing')
...
>>> new_obj = MyClass(2)
>>> MyClass.__name__ # имя
'MyClass'
>>> MyClass.__doc__ # документация
'Documentaion'
>>> new_obj.__dict__ # атрибуты экземпляра
{'value': 2}
\end{pcode}


\end{frame}

\section{Элементы ООП в Python}
\subsection*{}

\begin{frame}[fragile]\frametitle{ООП в Python}
Классы --- основной инструмент ООП в Python

\hfill

Основные принципы ООП (wiki):

\begin{itemize}
\item Полиморфизм --- свойство системы, позволяющее использовать объекты с одинаковым интерфейсом без информации о типе и внутренней структуре объекта

\item Инкапсуляция --- свойство системы, позволяющее объединить данные и методы, работающие с ними, в классе и ограничить к ним доступ.

\item Наследование --- свойство системы, позволяющее описать новый класс на основе уже существующего с частично или полностью заимствующейся функциональностью
\end{itemize}
\end{frame}

\begin{frame}[fragile]\frametitle{Наследование}

Класс может наследовать другие классы:
\begin{pcode}
>>> class Soldier:
...    def run(self):
...        print('run run')
...        
...    def shoot(self):
...        print('shoot shoot')
...    
>>> class Captain(Soldier):
...    def command(self, other_soldier):
...        other_soldier.run()
...
>>> s1 = Soldier()
>>> s2 = Captain()
>>> s2.command(s1)
'run run'
>>> s2.shoot()
'shoot shoot'
\end{pcode}
\end{frame}

\begin{frame}[fragile]\frametitle{Пример конфликтной ситуации}

\begin{pcode}
    >>> class A:
    ...    def get_attr(self):
    ...        return 'a'
    ...
    >>> class B:
    ...    def get_attr(self):
    ...        return 'b'
    ...
    >>> class C(A, B):
    ...     pass
    ...
    >>> obj = C()
    >>> obj.get_attr()
    'a'
\end{pcode}

\end{frame}

\begin{frame}[fragile]\frametitle{Разрешение конфликтов имён}

При обращении к атрибуту/методу экземпляра поиск ведётся:

\begin{enumerate}
\item В самом экземпляре
\item В классе экземпляра
\item По иерархии наследования
\end{enumerate}

\hfill

При множественном наследовании Python использует алогритм линеаризации C3 для определения вызываемого метода. Получить порядок:
\begin{pcode}
>>> C.__mro__
__main__.C, __main__.A, __main__.B, object
\end{pcode}

\end{frame}

\begin{frame}[fragile]\frametitle{Перегрузка конструктора наследника}
Если у класса-наследника определён метод \mintinline{python}{__init__}, конструкторы родителей необходимо вызывать явно:

\begin{pcode}
class Soldier:
    def __init__(self, speed=0):
        self.speed = speed
    
    def run(self):
        print('run ' * self.speed)
            
class Captain(Soldier):
    def __init__(self, speed=0, wisdom=0):
        Soldier.__init__(self, speed)
        self.wisdom = wisdom
        
    def command(self, other_soldier):
        for i in range(self.wisdom):
            other_soldier.run()
\end{pcode}

\end{frame}

\begin{frame}[fragile]\frametitle{Перегрузка конструктора наследника}
Другая реализация через  \mintinline{python}{super()}
(позволяет вызывать методы предков):
\begin{pcode}
class Soldier:
    def __init__(self, speed=0):
        self.speed = speed
    
    def run(self):
        print('run ' * self.speed)
            
class Captain(Soldier):
    def __init__(self, speed=0, wisdom=0):
        super().__init__(speed)
        self.wisdom = wisdom
        
    def command(self, other_soldier):
        for i in range(self.wisdom):
            other_soldier.run()
\end{pcode}

\end{frame}

\begin{frame}[fragile]\frametitle{Классы-примеси (mixin)}

    Если известно, что класс реализует некоторый интерфейс, можно использовать классы-примеси для его модификации: 
    
\begin{pcode}
    >>> def get_speed(self):
    ...     return self.speed
    >>> Soldier.get_speed = get_speed
    ...
    >>> class Speed_calculations_mixin:
    ...    def get_distance(self, time_sec):
    ...        return super().get_speed() * time_sec
    ...    
    ...    def get_time(self, distance):
    ...        return distance / super().get_speed()
    ...
    >>> class Major(Speed_calculations_mixin, Captain):
    ...    pass        
    \end{pcode}
\end{frame}

\begin{frame}[fragile]\frametitle{Абстрактные методы}

Пример абстрактного метода:

\begin{pcode}
class Quadrilateral:
    def __init__(self, coords):
        self.coords = coords
    def square():
        raise NotImplementedError("Implement this method")     
\end{pcode}
\end{frame}


\begin{frame}[fragile]\frametitle{Перегрузка операторов}

Перегрузка операторов реализуется за счёт написания \textit{специальных методов} (\mintinline{python}{__method__})

\hfill

Некоторые из них:

\begin{itemize}
\item \mintinline{python}{__init__} --- вызывается при создании нового экземпляра

\item \mintinline{python}{__add__} --- вызывается при сложении экземпляров

\item \mintinline{python}{__repr__} --- вызывается при выводе объекта

\item \mintinline{python}{__getattr__} --- вызывается при попытке прочитать
значение несуществующего атрибута

\item \mintinline{python}{__getitem__} --- взятие элемента по индексу

\end{itemize}

Полный список:
\href{https://docs.python.org/3/reference/datamodel.html\#special-method-names}{https://docs.python.org/3/reference/datamodel.html\#special-method-names}
\end{frame}

\begin{frame}[fragile]\frametitle{Пример перегрузки операторов}
\begin{pcode}
>>> class vector(list):
...    def __init__(self, some_list):
...        self.values = some_list
...    
...    def __add__(self, other_vector):
...        if len(self.values) == len(other_vector.values):
...            new_vector = vector([])
...            for elem in zip(self.values,
...                               other_vector.values):
...                new_vector.values.append(elem[0] + elem[1])
...            return new_vector
...        else:
...            raise TypeError('Wrong dimensions')
...            
...    def __getitem__(self, i):
...        if 0 <= i <= len(self.values):
...            return self.values[i]
\end{pcode}
\end{frame}

\begin{frame}[fragile]\frametitle{Сокрытие атрибутов}
В Python нет модификторов доступа к атрибутам и методам

\hfill

К внутренним атрибутам принято добавлять в начале символ подчёркивания:

\begin{pcode}
>>> class MyClass:
...    _important_attr = 32
\end{pcode}

\hfill

При добавлении двух подчёркиваний атрибут будет автоматически получать более сложное название:

\begin{pcode}
>>> class MyClass:
...    __very_important_attr = 32
...
>>> x = MyClass()
>>> x.__very_important_attr 
AttributeError: 'MyClass' object has no attribute ...
>>> x._MyClass__very_important_attr 
32
\end{pcode}


\end{frame}



\begin{frame}[fragile]\frametitle{Сокрытие атрибутов --- перегрузка операторов}
Для ограничения доступа к атрибутам можно перегрузить операторы  \mintinline{python}{__setattr__} и \mintinline{python}{__getattr__}  :
\begin{pcode}
>>> class A:
...    def __init__(self, value):
...        self.value = value
...
...    def __setattr__(self, name, value):
...        print('set atrribute', value)
...        self.__dict__[name] = value # словарь атрибутов
...
...    def __getattr__(self, name):
...        print('wrong attr')
...       
>>> a = A(123)
'set attribute 123'
>>> a.some_attr
'wrong attribute'
\end{pcode}


\end{frame}



\begin{frame}[fragile]\frametitle{Свойства}
Протокол свойств позволяет направлять операции чтения, записи и удаления для отдельных атрибутов отдельным функциям и методам

\hfill

\mintinline{python}{attribute = property(fget, fset, fdel, doc)}

\hfill
\begin{itemize}

\item \mintinline{python}{fget} --- функция, которая будет вызываться при попытке прочитать значение атрибута

\item \mintinline{python}{fset} --- функция, которая будет вызываться при попытке выполнить операцию присваивания

\item \mintinline{python}{fdel} --- функция, которая будет вызываться при попытке удалить атрибут. 

\item \mintinline{python}{doc} --- передается строка документирования с описанием атрибута

\end{itemize}

\end{frame}


\begin{frame}[fragile]\frametitle{Сокрытие атрибутов --- свойства}
\begin{pcode}
>>> class A:
...    def __init__(self, value):
...        self._value = value
...    
...    def get_val(self):
...        return self._value
...    
...    def set_val(self, value):
...        print('set atribute', value)
...        self._value = value
...        
...    value = property(get_val, set_val, None, "no doc")
...
>>> a = A(123)
>>> a.value = -1
'set atribute -1'
\end{pcode}
\end{frame}

\begin{frame}[fragile]\frametitle{Атрибут \texttt{\_\_slots\_\_}}
С помощью атрибута \mintinline{python}{__slots__} можно зафиксировать множество возможных атрибутов:

\begin{pcode}
>>> class Point:
...    __slots__ = ['x', 'y']
...    
...    def __init__(self, x, y):
...        self.x = x
...        self.y = y
...        
...    def get_r(self):
...        return (self.x ** 2 + self.y ** 2) ** 0.5
...
>>> m = Point(3, 4)
>>> m.z = 5
AttributeError: 'Point' object has no attribute 'z'
\end{pcode}

Экземпляры с \mintinline{python}{__slots__} требуют меньше памяти (если в \mintinline{python}{__slots__} нет \mintinline{python}{__dict__})
\end{frame}

\begin{frame}[fragile]\frametitle{Классы для реализации алгоритмов}
Основные сущности, с которыми мы будем работать:

\begin{itemize}
    \item Конкретное семейство алгоритмов --- класс
    
    \item Алгоритм из семейства с заданными параметрами --- экземпляр класса 
    
    \item Алгоритм обучения --- метод класса (\mintinline{python}{.fit})
    
    \item Получение предсказаний --- метод класса (\mintinline{python}{.predict})
\end{itemize}

\hfill

Какие преимущества у класса перед набором функций для реализации алгоритма?
\end{frame}

\begin{frame}[fragile]\frametitle{Библиотека scikit-learn}
Де-факто задала стандарт интерфейсов для алгоритмов машинного обучения.

\hfill

Сайт библиотеки: \href{http://scikit-learn.org/stable/}{http://scikit-learn.org/stable/}

\begin{itemize}
    \item[+] Огромное количество реализаций алгоритмов классификации, регрессии, кластеризации и т.д.
    \item[+] Отличная документация
    \item[--] Не всегда эффективные реализации
\end{itemize}
\end{frame}

\section*{}
\subsection*{}
\begin{frame}[fragile]\frametitle{Заключение}

Несколько важных замечаний:

\begin{itemize}
\item Функции, модули и классы являются объектами в Python

\item У функции может быть несколько аргументов

\item Модули выполняются при импортировании!

\item Класс и объект класса не одно и то же!

\item Частично скрывать атрибуты класса возможно, полностью нет

\end{itemize}
\end{frame}


\end{document}